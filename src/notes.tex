\part{Различные аспекты}

\section{Что такое PEP8}

python enhanced proposal — заявки на улучшение языка python), описывающий, какого стиля следует придерживаться при написании кода на C в реализации языка python

\section{Вызов в главной функции}
Что значит \pyth{if __name__=='__main__':}

Когда интерпретатор Python читает исходный файл, он исполняет весь найденный в нем код. Перед тем, как начать выполнять команды, он определяет несколько специальных переменных. Например, если интерпретатор запускает некоторый модуль (исходный файл) как основную программу, он присваивает специальной переменной \pyth{__name__} значение \pyth{"__main__"}. Если этот файл импортируется из другого модуля, переменной \pyth{__name__} будет присвоено имя этого модуля.

В случае с вашим сценарием, предположим, что код исполняется как основная функция, например:

\pyth{python threading_example.py}

После задания специальный переменных интерпретатор выполнит инструкцию import и загрузит указанные модули. Затем он проанализирует блок def, создаст объект-функцию и переменную под названием myfunction, которая будет указывать на этот объект.

Затем он прочтет инструкцию if, «поймёт», что \pyth{__name__} эквивалентен \pyth{"__main__"}, и выполнит указанный блок.

Одна из причин делать именно так – тот факт, что иногда вы пишете модуль (файл с расширением .py), предназначенный для непосредственного исполнения. Кроме того, он также может быть импортирован и использован из другого модуля. Производя подобную проверку, вы можете сделать так, что код будет исполняться только при условии, что данный модуль запущен как программа, и запретить исполнять его, если его хотят импортировать и использовать функции модуля отдельно.

Дополнительно см. \url{http://ibiblio.org/g2swap/byteofpython/read/module-name.html}.

\section{Импортирование модулей}
Что означает \pyth{threading_example} в данный момент импортируется из другого модуля?

Это означает, что кем-то в каком-либо файле .py (или в ходе интерактивной Python-сессии) используется выражение \pyth{import threading_example}. Противоположный этому случай – пользователь использует выражение \pyth{python threading_example.py} или \pyth{./threading_example.py}, и т. д.. В последнем случае, \pyth{threading_example.py} запущен как основная программа. В первом же случае он запущен как-то иначе (чтобы понять, ищите вызов вида \pyth{import threading_example}).

\section{Порядок вызова спец методов new, init, enter, exit, del}

Порядок вызова:

\begin{enumerate}
\item \pyth{__new__()};
\item \pyth{__init__()};
\item \pyth{__enter__()}, если вызывается инструкция with;
\item \pyth{__exit__()}, если вызывается инструкция with;
\item \pyth{__del__()}
\end{enumerate}

\section{Стандартные библиотеки python, работа с датами, регулярными выражениями}

Стандартные библиотеки python, работа с датами, регулярными выражениями

		\begin{itemize}
	
		\item unittest - поддерживает автоматизацию тестов, использование общего кода для настройки и завершения тестов, объединение тестов в группы, а также позволяет отделять тесты от фреймворка для вывода информации.
		\item subprocess - отвечает за выполнение следующих действий: порождение новых процессов, соединение c потоками стандартного ввода, стандартного вывода, стандартного вывода сообщений об ошибках и получение кодов возврата от этих процессов.
		\item fractions - предоставляет поддержку рациональных чисел.
		\item cmath – предоставляет функции для работы с комплексными числами.
		\item copy - поверхностное и глубокое копирование объектов.
		\item os.path - является вложенным модулем в модуль os, и реализует некоторые полезные функции на работы с путями.
		\item json - позволяет кодировать и декодировать данные в удобном формате. JSON (JavaScript Object Notation) - простой формат обмена данными, основанный на подмножестве синтаксиса JavaScript.
		\item calendar - позволяет напечатать себе календарик (а также содержит некоторые другие полезные функции для работы с календарями).
		\item os - предоставляет множество функций для работы с операционной системой, причём их поведение, как правило, не зависит от ОС, поэтому программы остаются переносимыми.
		\item pickle - реализует мощный алгоритм сериализации и десериализации объектов Python. "Pickling" - процесс преобразования объекта Python в поток байтов, а "unpickling" - обратная операция, в результате которой поток байтов преобразуется обратно в Python-объект. Так как поток байтов легко можно записать в файл, модуль pickle широко применяется для сохранения и загрузки сложных объектов в Python.
		\item datetime - предоставляет классы для обработки времени и даты разными способами. Поддерживается и стандартный способ представления времени, однако больший упор сделан на простоту манипулирования датой, временем и их частями.
		\item array - определяет массивы в python. Массивы очень похожи на списки, но с ограничением на тип данных и размер каждого элемента.
		\item Time - модуль для работы со временем в Python.
		\item. sys - обеспечивает доступ к некоторым переменным и функциям, взаимодействующим с интерпретатором python.
		\item math – один из наиважнейших в Python. Этот модуль предоставляет обширный функционал для работы с числами.
		\item регулярное выражение — это последовательность символов, используемая для поиска и замены текста в строке или файле. В Python для работы с регулярными выражениями есть модуль re. Регулярные выражения используют два типа символов: специальные символы: как следует из названия, у этих символов есть специальные значения, например, * означает «любой символ»; литералы (например: a, b, 1, 2 и т. д.).
	\end{itemize}

\section{Принципы ООП}

\begin{itemize}
\item Абстракция данных - Абстрагирование означает выделение значимой информации и исключение из рассмотрения незначимой. 
\item Инкапсуляция — свойство системы, позволяющее объединить данные и методы, работающие с ними, в классе.
\item Наследование — свойство системы, позволяющее описать новый класс на основе уже существующего с частично или полностью заимствующейся функциональностью. Класс, от которого производится наследование, называется базовым, родительским или суперклассом. Новый класс — потомком, наследником, дочерним или производным классом.
\item Полиморфизм подтипов (в ООП называемый просто «полиморфизмом») — свойство системы, позволяющее использовать объекты с одинаковым интерфейсом без информации о типе и внутренней структуре объекта.
\item Класс - универсальный, комплексный тип данных, состоящий из тематически единого набора «полей» (переменных более элементарных типов) и «методов» (функций для работы с этими полями), то есть он является моделью информационной сущности с внутренним и внешним интерфейсами для оперирования своим содержимым (значениями полей). 
\item Объект - Сущность в адресном пространстве вычислительной системы, появляющаяся при создании экземпляра класса.
\end{itemize}

Согласно Алану Кэю — автору языка программирования Smalltalk — объектно-ориентированным может называться язык, построенный с учетом следующих принципов:

\begin{itemize}
\item Все данные представляются объектами
\item Программа является набором взаимодействующих объектов, посылающих друг другу сообщения
\item Каждый объект имеет собственную часть памяти и может иметь в составе другие объекты
\item Каждый объект имеет тип
\item Объекты одного типа могут принимать одни и те же сообщения (и выполнять одни и те же действия)
\end{itemize}