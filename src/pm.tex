\chapter{Управление проектами}

\section{Менеджмент}
Искусство управления при ограниченных ресурсах.

\section{Этапы управления}

\subsectoin{Формирование требований}
\subsectoin{Сбор команды}
\subsectoin{Формирование ТЗ}
\subsectoin{Формирование системы контроля}
\subsection{Формирование критерии успеха}
\subsection{Формирование процесса разработки}

\section{Признаки операционной деятельности}

\subsection{Повторяемость}
\subsection{Регламенитруемость}

\section{Программа проекта, отличия}
Программа проекта включает в себя обратную связь для учета различных факторов. Как следствие может меняться для достижения цели проекта. Например, остановкой одного подпроекта и запуском другого.

\section{Признаки проекта}

\subsection{Уникальность}
\subsection{Ограниченность по времени}

\section{Продукт проекта}

\section{Закон Лермана}
Любую задачу можно решить при наличии необходимого количества времени и денежных средств.

\section{Классификация проектов}

Важно правильно адекватно распределить ресурсы.

\begin{table}[h!]
\begin{tabular}{|l|l|l|}
\hline
              & Проекты развития и адаптации (внутренние)   & Контрактные проекты (внешние) \\ \hline
Краткосрочные & Не важный, не самые квалифицированные спец. &                               \\ \hline
Среднесрочные &                                             &                               \\ \hline
Долгосрочные  &                                             & Самый важный, лучший персонал \\ \hline
\end{tabular}
\end{table}

\section{Жизненный цикл проекта}

Обычно имеет 4 фазы - типовая схема. Нужна для контроля на каждом из этапов. Применяется термит веха. Так же нужна для того, что бы команда понимала что делать. Передача жизненного цикла проекта - важно для передачи дел.

\subsection{Концепция}
Думаем
\subsection{Разработки}
Планы, надежды
\subsection{Реализация}
\subsection{Завершение}

\section{Роли}

\begin{itemsize}
\item Куратор проекта - разделяет ответственность с руководителем, делигировать полномочия
\item Руководитель проекта
\item Заказчик
\item Инвестор
\item Функциональный заказчик
\item Куратор - связующее звено, соблюдает баланс между хотелками остальных.
\end{itemsize}

\section{Организацонные структуры}

\begin{itemsize}
\item Функциональная - проектом занимается какой-то один отдел.
\item Командая - сборная  солянка из разных отделов, занимающихся только этим проектом.
\item Матричная - люди из разных подразделений тратят какое-то количество времени.
\end{itemsize}

\section{Процесс управления}
Описывается группами процессами. Все описывается планами: временными, финансами, ресурсами.

\subsection{Группа процессов инициация} 
Цели, критерии успешности.
\subsection{Группа процессов планирования}
Календарный план. Выявление рисков
\subsection{Группа процессов реализации}
\subsection{Группа процессов контроля качества}
Нужна для повышения качества.
\subsection{Группа процессов формирования архива}

\section{Сетевые диаграммы}
Временные резервы. Критический путь проекта. 
