\chapter{Управление проектами}

\section{Менеджмент}
Искусство управления при ограниченных ресурсах.

\section{Этапы управления}

\subsectoin{Формирование требований}
\subsectoin{Сбор команды}
\subsectoin{Формирование ТЗ}
\subsectoin{Формирование системы контроля}
\subsection{Формирование критерии успеха}
\subsection{Формирование процесса разработки}

\section{Признаки операционной деятельности}

\subsection{Повторяемость}
\subsection{Регламенитруемость}

\section{Признаки проекта}

\subsection{Уникальность}
\subsection{Ограниченность по времени}

\section{Продукт проекта}

\section{Закон Лермана}
Любую задачу можно решить при наличии необходимого количества времени и денежных средств.

\section{Классификация проектов}

Важно правильно адекватно распределить ресурсы.

\begin{table}[h!]
\begin{tabular}{|l|l|l|}
\hline
              & Проекты развития и адаптации (внутренние)   & Контрактные проекты (внешние) \\ \hline
Краткосрочные & Не важный, не самые квалифицированные спец. &                               \\ \hline
Среднесрочные &                                             &                               \\ \hline
Долгосрочные  &                                             & Самый важный, лучший персонал \\ \hline
\end{tabular}
\end{table}

\section{Жизненный цикл проекта}

Обычно имеет 4 фазы - типовая схема. Нужна для контроля на каждом из этапов. Применяется термит 'веха'. Так же нужна для того, что бы команда понимала что делать. Передача жизненного цикла проекта - важно для передачи дел.

\subsection{Концепция}
Думаем
\subsection{Разработки}
Планы, надежды
\subsection{Реализация}
\subsection{Завершение}
