\part{Общие задачи}

	\section{Вычисление факториала \label{tasks:fac}}
	Задача имеет множество решений. По этой причине реализуем хотя бы несколько с проверкой ответа из библиотеки python..
	
\linenumbers
\inputpython{factorials.py}{1}{100}
\nolinenumbers

	\section{Числа Фибоначчи \label{tasks:fib}}
	
	\section{Определение простых чисел \label{tasks:simple} }
	
\linenumbers
\inputpython{simple-nums.py}{1}{100}
\nolinenumbers
	
	\section{Найти наименьший общий делитель двух чисел (НОД)}
	\textbf{Алгоритм Евклида}:
	\begin{enumerate}
\item Большее число делим на меньшее.
\item Если делится без остатка, то меньшее число и есть НОД (следует выйти из цикла).
\item Если есть остаток, то большее число заменяем на остаток от деления.
\item Переходим к пункту 1.

	\end{enumerate}
	
	\section{Найти наибольшее общее кратное двух чисел (НОК)}
	
	\section{Работа с датами}
	\textbf{Задача:}
	
	Дана таблица в базе данных SQL состоящая из колонок: имя трудяги, дата устройства на работу, дата увольнения\dots необходимо выбрать всех работников которые:
	\begin{enumerate}
	\item отработали в текущем месяце больше 20 дней
	\item отработали за все время больше 30 дней
	\item уволились меньше чем через 60 дней
	\end{enumerate}
	
	\textbf{Решение:}
	
	
	
	\section{Шифр Цезаря}
Шифр Цезаря заключается в замене каждого символа входной строки на символ, находящийся на несколько позиций левее или правее его в алфавите.

Для всех символов сдвиг один и тот же. Сдвиг циклический, т.е. если к последнему символу алфавита применить единичный сдвиг, то он заменится на первый символ, и наоборот.

Напишите программу, которая шифрует текст шифром Цезаря.

Используемый алфавит − пробел и малые символы латинского алфавита: ' abcdefghijklmnopqrstuvwxyz'

\textbf{Формат ввода:}
На первой строке указывается используемый сдвиг шифрования: целое число. Положительное число соответствует сдвигу вправо. На второй строке указывается непустая фраза для шифрования. Ведущие и завершающие пробелы не учитывать.

\textbf{Формат вывода:}
Единственная строка, в которой записана фраза: Result: "..." , где вместо многоточия внутри кавычек записана зашифрованная последовательность.

\textbf{Примеры}

\begin{lstlisting}
Sample Input 1:
3
i am caesar
Sample Output 1:
Result: "lcdpcfdhvdu"
Sample Input 2:
-2
az
Sample Output 2:
Result: "zx"
Sample Input 3:
27
abc
Sample Output 3:
Result: "abc"
\end{lstlisting}

Решение:

\linenumbers
\inputpython{task1.py}{1}{100}
\nolinenumbers

\section{Парсинг лог файла}

\linenumbers
\inputpython{parse-log.py}{1}{100}
\nolinenumbers	

\section{Задача со столиками}

\subsection{Условие}

В вашем ресторане есть набор столов разных размеров: на каждом столе могут разместиться 2, 3, 4, 5 или 6 человек. Клиенты прибывают в одиночку или группами, до 6 человек. Клиенты в данной группе должны находиться вместе за одним столом, поэтому вы можете направить группу только к столу, который может вместить их всех. Если нет таблицы с необходимым количеством пустых стульев, группе приходится ждать в очереди.

После размещения группа не может изменить таблицу, т. Е. Вы не можете переместить группу из одной таблицы в другую, чтобы освободить место для новых клиентов.

Группы клиентов должны обслуживаться в порядке прибытия, за одним исключением: если на столе достаточно места для небольшой группы, прибывающей позже, вы можете разместить их перед более крупной группой (группами) в очереди. Например, если есть группа из шести человек, ожидающая стол на шесть мест, и есть группа из двух человек, которые стоят в очереди или прибывают, вы можете отправить их прямо к столу с двумя пустыми стульями.

Группы могут совместно использовать столы, однако если в то же время у вас есть пустой стол с необходимым количеством стульев и достаточным количеством пустых стульев за большим, вы всегда должны размещать своих клиентов за пустым столом, а не частично сидящими. , даже если пустая таблица больше, чем размер группы.

Конечно, система предполагает, что любая большая группа может устать от того, что меньшие группы приходят и опережают их столы, а затем решают уйти, что будет означать, что они покидают очередь без обслуживания.

Пожалуйста, заполните класс RestManager соответствующими структурами данных и реализуйте его конструктор и три открытых метода. Также рекомендуется изменить другие классы (чтобы мы могли их протестировать) и добавить новые методы по вашему желанию.

\subsection{Решение}




