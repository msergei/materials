\part{Базовые концепции}

\section{Разделяй и властвуй \label{idea:diff}}

\section{Не повторяйся (DRY) \label{idea:dry}}

\textbf{DRY} - \textbf{Do not repeat yourself} - \textbf{Не повторяйся} - концепция программирования при которой код пишется только один раз, для реализации этого принципа применяются различные методологии программирования, например, \textbf{ООП} (см. \ref{method:oop}), а так же функции и шаблоны (см. \ref{idea:pattern}).

\section{Рекурсия \label{idea:recur}}

\textbf{Рекурсия} - это абстракция, которая означает повторение какого-то действия с измененными входными данными. Применительно к теории алгоритмов, определяют рекурсивную функцию. 

\textbf{Рекурсивная функция} - это алгоритм, который вызывает сам себя, тем самым образуя стек вызовов, с измененными аргументами. Что бы функция не поучилась бесконечной, то определяют базовый случай, который завершает работу функции и всего стека вызовов.

В общем случае алгоритм функции состоит из двух условий:
\begin{enumerate}
\item Инструкции вызова этой же функции с передачей измененных аргументов;
\item Базовый случай, который завершает работу функции.
\end{enumerate}

Рекурсия не уменьшает время работы кода, она лишь может уменьшить время работы программиста! Отлично подходит под концепцию DRY. Пример реализации задача вычисления факториала (см. \ref{tasks:fac}), чисел Фибонначи (см \ref{tasks:fib}) и определение простых чисел (см. \ref{tasks:simple}). 

\section{Шаблон проектирования или паттерн \label{idea:pattern}}

\textbf{Шаблон проектирования} или \textbf{паттерн} (англ. \textbf{design pattern}) в разработке программного обеспечения — повторяемая архитектурная конструкция, представляющая собой решение проблемы проектирования в рамках некоторого часто возникающего контекста.