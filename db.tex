\part{Databases}

\chapter{Основные понятия}

\section{Теорема CAP}

Теорема CAP (известная также как теорема Брюера) — эвристическое утверждение о том, что в любой реализации распределённых вычислений возможно обеспечить не более двух из трёх следующих свойств:
\begin{itemize}

 \item согласованность данных (англ. consistency) — во всех вычислительных узлах в один момент времени данные не противоречат друг другу;
\item доступность (англ. availability) — любой запрос к распределённой системе завершается корректным откликом, однако без гарантии, что ответы всех узлов системы совпадают;
\item устойчивость к разделению (англ. partition tolerance) — расщепление распределённой системы на несколько изолированных секций не приводит к некорректности отклика от каждой из секций.

\end{itemize}

\section{Чем отличается транзакция от запроса в контексте БД}
Транзакция - это несколько последовательных запросов в БД, если хотя бы одна из них завершается, то вся транзакция отменяется (rolling back).

\section{Что такое ACID}
\section{Есть ли транзакция в Redis?}

\chapter{Масштабируемость}

\section{Что такое согласованность данных?}

\section{Алгоритмы разрешения конфликтов при репликации}
\subsection{Quorum}
\subsection{Векторные часы}

\section{Шардирование}
\section{Оптимистичные блокировки и пессимистичные?}

\section{Реляционная теория}
\subsection{Что такое транзитивная завивсимость?}
\subsection{Нормальные формы?}
\begin{enumerate}
    \item 1ая;
    \item 2ая;
    \item 3ая;
    \item Бойса-Кодда;
    \item 4ая;
    \item Доменная форма;
    \item 5ая;
    \item 6ая;
\end{enumerate}
\subsection{Ограничение целостности базы данных?}


\chapter{PostgreSQL}

\section{Архитектура posgres}
\subsection{Внутренние компоненты}
\subsection{Что такое autovacuum?}
\section{Что такое план запроса?}

\section{Индексы в postgres}

\subsection{Типы индексов}
\subsubsection{b-tree}
\subsubsection{hash}
\subsubsection{gin}
\subsubsection{gist}
\subsubsection{sp-gist}
\subsubsection{brin}

\section{Что такое include индексы?}
\section{Как посмотреть индексы для таблицы?}
\section{Всегда ли postgres использует индексы?}

\section{Типы данных Redis}

\subsection{Строки (strings)}
Базовый тип данных Redis. Строки в Redis бинарно-безопасны, могут использоваться так же как числа, ограничены размером 512 Мб.
\subsection{Списки (lists)}
Классические списки строк, упорядоченные в порядке вставки, которая возможна как со стороны головы, так и со стороны хвоста списка. Максимальное количество элементов - 232 - 1.
\subsection{Множества (sets)}    
Множества строк в математическом понимании: не упорядочены, поддерживают операции вставки, проверки вхождения элемента, пересечения и разницы множеств. Максимальное количество элементов - 232 - 1.
\subsection{Хеш-таблицы (hashes)}    
Классические хеш-таблицы или ассоциативные массивы. Максимальное количество пар «ключ-значение» - 232 - 1.
\subsection{Упорядоченные множества (sorted sets)}
Упорядоченное множество отличается от обычного тем, что его элементы упорядочены по особому параметру «score».